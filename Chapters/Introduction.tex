\chapter*{Introduction générale}

\addcontentsline{toc}{chapter}{Introduction générale }

\label{chap: Conclusion générale} 


\hspace{\parindent}La transformation digitale est devenue un impératif stratégique pour le secteur bancaire, sous l'impulsion de régulations et de normes établies, notamment par le Groupement Professionnel des Banques du Maroc (GPBM). Dans ce cadre, B3G, une entreprise spécialisée dans le développement de solutions digitales pour les banques, a conçu la suite Fawri-Flow pour moderniser et optimiser les opérations bancaires.

Afin de répondre aux besoins croissants de cette transformation digitale et de faciliter la gestion de contenus et processus métiers complexes, il est apparu nécessaire de développer un CMS Low-Code. Cette démarche a conduit à la réalisation de notre projet de fin d’études, portant sur le développement de Fawri-CMS, une solution innovante et flexible.

Fawri-CMS vise à simplifier le processus de développement d'applications grâce à une approche "Low-Code" et à intégrer les processus métier de manière transparente. Cela permet de réduire significativement le temps et les ressources nécessaires au développement. La solution permet également à un plus grand nombre d'utilisateurs de créer des applications adaptées à leurs besoins spécifiques, ce qui lui a valu une reconnaissance notable lors du \textbf{"Saudi No Code Innovation Summit"}.

Ce rapport présente les différentes étapes de la réalisation de cette solution, en mettant en lumière les choix et décisions prises tout au long du projet, ainsi que les résultats obtenus. Il est structuré en quatre chapitres principaux :
\begin{itemize}
\item \textbf{Présentation Générale}

Ce chapitre contextualise le projet, présentant les objectifs du stage, l'entreprise B3G, et l'application à développer. Il aborde également les enjeux de la transformation digitale, les concepts de Low-Code et de CMS, ainsi que les choix technologiques et méthodologiques retenus pour mener à bien le projet.

\item \textbf{Analyse et Conception}

Ce chapitre se concentre sur l'analyse détaillée des besoins fonctionnels de l'application. Il inclut la modélisation conceptuelle des différents domaines fonctionnels, identifiant les classes, les relations, et les processus métier impliqués.

\item \textbf{Outils, Méthodologies et Architecture}

Ici, l'accent est mis sur les outils, les méthodologies et l'architecture utilisés lors du développement de l'application. Le chapitre décrit les technologies employées, telles que le framework .NET et ABP, ainsi que les principes de conception comme TDD, SOLID, et DDD. L'architecture logicielle est expliquée en soulignant les choix architecturaux et les bonnes pratiques adoptées.

\item \textbf{Implémentation et Réalisation}

Ce dernier chapitre se focalise sur l'implémentation de l'application. Il détaille les étapes de développement, les fonctionnalités implémentées, les tests réalisés, et les résultats obtenus. Une rétrospective sur le déroulement du projet et les résultats est également incluse.
\end{itemize}
Ce rapport a pour but de fournir une vue d'ensemble complète de la création et du développement de Fawri-CMS, tout en illustrant les défis rencontrés et les solutions apportées pour les surmonter.