\chapter*{Résumé}
\addcontentsline{toc}{chapter}{Résumé}

\hspace{\parindent} Ces dernières années, le secteur bancaire a connu une poussée significative vers la transformation numérique, nécessitant des solutions innovantes pour répondre aux besoins technologiques en constante évolution. Ce rapport présente le développement de Fawri CMS, une solution numérique de pointe adaptée à l’industrie bancaire, élaborée au sein de l’entreprise B3G dans le cadre de mon stage de fin d’études. Fawri-CMS simplifie le développement d’applications grâce à une approche ”Low Code”, permettant aux développeurs de créer des applications robustes avec un minimum de codage manuel tout en intégrant de manière transparente des processus métier complexes. Le système propose des fonctionnalités avancées de gestion de contenu (CMS) telles que la création, la gestion et la publication de contenu, conçues pour répondre à des normes de sécurité et de conformité strictes. Le projet a suivi une méthodologie de gestion de projet agile, en particulier le cadre SCRUM, facilitant le développement itératif et les boucles de rétroaction régulières. Le parcours a débuté par une phase d’analyse et de conception approfondie utilisant des diagrammes UML pour modéliser l’architecture du système, suivie de la conception technique et de la configuration de l’environnement de développement, incluant SQL Server et .NET Core.
 La phase de développement a impliqué l’application des principes de la conception pilotée par le domaine (DDD) pour garantir la résilience et la maintenabilité du système. Un aspect remarquable a été la mise en œuvre de la fonctionnalité de création de formulaires, permettant de glisser-déposer des composants, en utilisant JavaScript. Ce composant a ensuite été publié sur npm et intégré de manière transparente à notre solution .NET, en harmonie avec les autres fonctionnalités du CMS.

\noindent\rule[2pt]{\textwidth}{0.5pt}

{\textbf{Mots clés :}}
Secteur bancaire, transformation digitale, approche Low Code, business processes, CMS, sécurité, méthodologie agile, UML, SQL Server, .NET Core, DDD, JavaScript, npm.
\\
\noindent\rule[2pt]{\textwidth}{0.5pt}