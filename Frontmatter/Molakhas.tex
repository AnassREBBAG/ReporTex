
\chapter*{\RL{ملخص}}
\addcontentsline{toc}{chapter}{Résumé en Arabe}

\begin{RLtext}
    في السنوات الأخيرة، شهد قطاع البنوك دفعة كبيرة نحو التحول الرقمي، مما يتطلب حلولًا مبتكرة لتلبية الاحتياجات التكنولوجية المتطورة. يبرز هذا التقرير تطوير \LR{Fawri-CMS} الحل الرقمي المتقدم المصمم خصيصًا لصناعة البنوك والذي تم تطويره داخل شركة \LR{B3G} خلال فترة تدريبي النهائية.

    يبسط \LR{Fawri-CMS} تطوير التطبيقات من خلال منهجية \LR{"Low-Code"} مما يمكن المطورين من بناء تطبيقات قوية مع الحد الأدنى من البرمجة اليدوية مع دمج سلس للعمليات التجارية المعقدة. يتميز النظام بميزات متقدمة لنظام إدارة المحتوى \LR{(CMS)} بما في ذلك إنشاء المحتوى وإدارته ونشره، مصممة بدقة لتلبية معايير الأمان والامتثال الصارمة.

    اعتمد المشروع منهجية إدارة مشاريع البرمجيات الرشيقة والمضبوطة \LR{SCRUM} مما يسهل التطوير التكراري وحلقات الردود الدورية. بدأت الرحلة بمرحلة تحليل وتصميم دقيقة باستخدام مخططات \LR{UML} لنمذجة هندسة النظام، تلاها التصميم الفني وإعداد بيئة التطوير، بما في ذلك \LR{SQL Server} و \LR{.NET Core}.

    شملت مرحلة التطوير تنفيذ مبادئ التصميم القائم على المجال \LR{DDD} لضمان قوة النظام وقابليته للصيانة. كان جزءاً هاماً من وظائف بناء النماذج، التي تتيح إسقاط المكونات، مطورًا باستخدام \LR{JavaScript}. ثم يتم تقييد هذا المكون على \LR{npm} ودمجه ببساطة مع الميزات الأخرى لنظام إدارة المحتوى.


\end{RLtext}

\noindent\rule[2pt]{\textwidth}{0.5pt}

\begin{RLtext}

    {\textbf{الكلمات المفتاحية}}
    قطاع البنوك، التحول الرقمي، نهج "الكود المنخفض"، عمليات الأعمال، نظام إدارة المحتوى.

    \\

\end{RLtext}

\noindent\rule[2pt]{\textwidth}{0.5pt}