\chapter{Glossaire}
\label{chap:Glossary}


\hspace{\parindent}\textbf{CMS:} CMS est l’acronyme de Content Management System, c’est-à-dire système de gestion de contenu. Il s’agit d’un logiciel en ligne grâce auquel il est possible de créer, de gérer et de modifier facilement un site web, sans avoir besoin de connaissances techniques en langage informatique.\\

\textbf{Low Code:} c'est une approche de développement logiciel qui permet de créer des applications rapidement avec un minimum de programmation manuelle. Cette méthode utilise des outils de développement visuels pour permettre aux utilisateurs de créer des applications en drag-and-drop, en cliquant sur des éléments de l’interface graphique et en utilisant des blocs de code préconstruits.\\

\textbf{DDD:} (Domain-Driven Design) est une approche de conception logicielle centrée sur la compréhension et la modélisation approfondies du domaine d'application. Elle met en avant l'utilisation d'un modèle de domaine précis et riche, un langage ubiquitaire partagé par les développeurs et les experts du domaine, et des contextes délimités pour organiser le modèle en sous-domaines cohérents.\\


\textbf{L'approche TDD :} (Test-Driven Development) est une approche de développement logiciel qui consiste à écrire des tests automatisés avant d'écrire le code fonctionnel. Le processus commence par la rédaction d'un test unitaire qui spécifie et valide ce que le code doit accomplir. Ensuite, le développeur écrit le code minimal nécessaire pour faire passer ce test. Une fois le test réussi, le code est refactorisé pour améliorer sa structure tout en s'assurant que les tests continuent à passer.\\


\pagebreak